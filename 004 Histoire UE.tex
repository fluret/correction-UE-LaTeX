\documentclass{report}%           autres choix : report, book

\usepackage[utf8]{inputenc}%       encodage du fichier source
\usepackage[T1]{fontenc}%          gestion des accents (pour les pdf)
\usepackage[french]{babel}%      rajouter éventuellement english, greek, etc.
\usepackage{textcomp}%             caractères additionnels
\usepackage{amsmath,amssymb}%      pour les maths
\usepackage{lmodern}%              remplacer éventuellement par txfonts, fourier, etc.
\usepackage[a4paper,hmargin=1.5cm,vmargin=2cm]{geometry}%    taille correcte du papier
\usepackage{graphicx}%             pour inclure des images
\usepackage{xcolor}%               pour gérer les couleurs
\usepackage{microtype}%            améliorations typographiques
\usepackage{lastpage}
\usepackage{fancyhdr}
%avec cette commande on est obligé de remettre un
%\thispagestyle{fancy} après chaque chapitre, cette commande permet de différencier les chapitres des autres pages
%\pagestyle{fancy}
%avec la commande suivante, cela n'est plus nécessaire
\pagestyle{fancyplain}

\usepackage[hidelinks]{hyperref}%             gestion des hyperliens
\hypersetup{pdfstartview=XYZ}%     zoom par défaut
\author{Frédéric LURET}
\title{Histoire de l’union Européenne}
\date{\today}
%\pagestyle{headings}

%personnalisation des en-tête et pied de page
\setlength{\headheight}{0.5cm}
\fancyhead[L]{\leftmark}
\fancyhead[R]{}

\fancyfoot[L]{\bfseries Frédéric LURET}
\fancyfoot[C]{\vspace{0.5em} \today}
\fancyfoot[R]{\thepage\ sur \pageref{LastPage}}


\renewcommand{\footrule}{{\color{blue}\dotfill}}
\renewcommand{\headrule}{{\color{blue}\dotfill}}


\begin{document}
\maketitle

\tableofcontents
\thispagestyle{empty}
\newpage
\chapter{Prémices}

\section{L'Europe avant l'Europe, Histoire de l'ancien continent}
Le terme Europe n'était pas fréquemment employé avant la Renaissance des XVe et XVIe siècles. « Europe » était bien le terme désignant le continent nord européen (au côté des deux autres continents, l'Asie et l'Afrique), mais on l'employait avec un sens strictement géographique, et non politique ou culturel. On employait plus fréquemment le terme Occident (par opposition à l'Orient byzantin), l'Amérique n'étant pas encore découverte. Le terme chrétienté englobait l'Orient chrétien.
\subsection{Empire romain}
	\begin{itemize}
		\item Celui-ci englobait le bassin méditerranéen à partir du Ier siècle av. J.‑C. Après le partage de 395, l'Empire romain fut séparé entre l'Empire romain d'Occident et l'Empire romain d'Orient.
		\item L'empire d'Occident ne survécut pas à la période des Grandes invasions, tandis que l'empire d'Orient survécut. Nous le connaissons sous le nom d'Empire byzantin.
	\end{itemize}
\subsection{Renaissance carolingienne}
	\begin{itemize}
		\item Après plus de deux siècles de décadence en Occident, on chercha à restaurer les bases de la civilisation. Le modèle en fut celui de l'empire d'Occident.
		\item Après l'arrêt des invasions arabes (732), Pépin le Bref restaura un pouvoir stable en accord avec l'Église catholique. Son fils Charlemagne établit un pouvoir politique par des conquêtes territoriales. Il se considérait comme le successeur des empereurs romains. L'empire carolingien préfigura l'unité de l'Europe actuelle.
		\item Les problèmes de succession aboutirent au démantèlement de l'Empire carolingien après la mort de l'empereur Louis le Pieux (partage de Verdun de 843) et les invasions normandes, sarrasines et hongroises du IXe siècle dévastèrent une grande partie de l'ancien empire.
	\end{itemize}
\subsection{Renaissance ottonienne}
	\begin{itemize}
		\item L'empire d'Occident survécut dans l'ancienne Germanie, sous la forme du Saint-Empire romain germanique, avec les empereurs ottoniens.
		\item Les ordres religieux permirent un renouveau de civilisation à partir des années 920-950 environ : ordre de Cluny en Francie, qui s'étendit à une grande partie de l'Europe au XIe siècle, et abbaye de Gorze en Lotharingie puis dans le Saint-Empire romain germanique.
		\item Cette période est quelquefois appelée renaissance ottonienne, en référence aux empereurs ottoniens et à l'ordre de Cluny, les prémices de l'amitié franco-allemande étant représentées par l'amitié entre Otton III et Gerbert d'Aurillac (futur pape Sylvestre II).
		\item Loin d'être une période de « terreurs » (prétendues terreurs de l'an mil), l'an mil correspond plutôt à une renaissance (Pierre Riché).
	\end{itemize}
\subsection{Renaissance du XIIe siècle}
	\begin{itemize}
		\item Les échanges avec la civilisation arabo-musulmane, alors très en avance par rapport à l'Occident chrétien, engendrèrent une nouvelle période de renouveau à la suite des premières croisades. En pleine période d'essor urbain, le renouveau du savoir se traduisit par l'ouverture d'universités dans quelques grandes villes européennes.
		\item De nouveaux ordres religieux apparurent (ordre de Cîteaux), puis l'extension des villes donna naissance aux ordres franciscains et dominicains au XIIIe siècle.
		\item L'unité de l'Europe était donc surtout religieuse et culturelle. La querelle, entre guelfes et gibelins aux XIIe et XIIIe siècles, témoigne d'une recherche d'organisation politique\footnote{Jean-Charles-Léonard Simonde de Sismondi, Histoire des Républiques italiennes du Moyen Âge, Université d'Oxford, Treuttel et Würtz, 1818}. Le pouvoir politique se démarqua de l'Église à partir de Philippe le Bel (début du XIVe siècle).
		\item La peste noire et la guerre de Cent Ans interrompirent cet élan.
	\end{itemize}
\subsection{Renaissance}
	\begin{itemize}
		\item Érasme avançait déjà dans Plaidoyer pour la paix l'idée d'un grand ensemble européen. Érasme désirait que celui-ci contînt les États chrétiens et qu'il reposât sur des valeurs de tolérance et de paix.
		\item Sous le règne de Charles Quint, empereur du Saint-Empire romain germanique (première partie du XVIe siècle), les Habsbourg dominent l'Autriche, l'Espagne, la Franche-Comté, Milan, Naples et les Pays-Bas. Ils ont façonné l'Union européenne en l'unifiant politiquement par des guerres, mais aussi par des mariages.
		\item Aux XVe et XVIe siècles, l'Europe obtint la suprématie mondiale grâce aux grandes découvertes : c'est le début d'une « protomondialisation », dont la caractéristique est le commerce triangulaire.
		\item Aux Traités de Westphalie en 1648, pour la première fois se retrouvent autour d'une table de négociation les grands États d'Europe. Et c'est la première fois aussi que sont définies les relations entre les États dans le respect de la souveraineté de chacun. Il n'est plus question comme au Moyen Âge d'une chrétienté occidentale unie autour d'une foi commune sous la haute autorité du souverain pontife. Chaque monarque est désormais maître chez lui, y compris en matière religieuse (principe Cujus regio, ejus religio).
	\end{itemize}
\subsection{Lumières et XIXe siècle}
	\begin{itemize}
		\item Le siècle des Lumières vit des échanges culturels transcendant les frontières : les Lumières françaises avaient leur équivalent avec l'« Enlightenment » anglais, l'Aufklärung allemand, l'« Illuminismo » (en italien), l'« Ilustración » (en espagnol). Les musiciens commençaient à parcourir l'Europe (Haendel, Mozart).
		\item La Révolution française avait l'ambition d'étendre les valeurs révolutionnaires à toute l'Europe.
		\item Napoléon étendit temporairement la domination française sur la plus grande partie de l'Europe, avec des territoires sous administration française (130 départements) ou des royaumes sous influence française, et propagea certaines des valeurs révolutionnaires. Il poussa ses armées jusqu'à Moscou lors de la campagne de Russie.
		\item Le Congrès de Vienne consacre en 1815 la création d'un ordre européen où les relations entre États sont contractualisées, avec un « système des congrès » dans lequel les diplomates du continent se réunissent régulièrement. La Sainte-Alliance est le premier traité postulant une unique nation en Europe, signé par l'ensemble des pays du continent, sauf l'Angleterre et le Vatican.
	\end{itemize}
\subsection{XXe siècle}
	\begin{itemize}
		\item L’Empire austro-hongrois et l’Empire allemand disparurent définitivement au traité de Versailles (1918).
		\item Pendant la Seconde Guerre mondiale, le projet d'intégration de l'Europe sous domination nazie sous Hitler chercha à imposer la vision d'un Empire européen mené par l'Allemagne nazie (Troisième Reich), destiné à durer mille ans (promesses d'un Ordre nouveau).
		\item Les dégâts humains et économiques de la guerre imposent l'idée d'une nécessaire pacification des relations sur le continent.
		\item Après la Seconde Guerre mondiale, les États-Unis appuient la création d'une union économique et assurent la défense militaire de l'Europe de l'Ouest face au Bloc de l'Est.
	\end{itemize}
\section{Une grande idée : les États-Unis d'Europe}
L'idée européenne s'est construite progressivement à partir du XVIe siècle. Dès la Renaissance, en 1526, dans son ouvrage Des conflits européens et de la guerre turque, l’humaniste espagnol Luis Vives préconise une union des royaumes d’Europe contre les Turcs. L’Espagne est alors à son apogée et doit mener presque seule des combats contre les Turcs qui s’imposent en Méditerranée.
Les penseurs des Lumières évoquaient déjà cette idée. En 1713, suite à la guerre de Succession d’Espagne, longue et sanglante guerre paneuropéenne (1701-1713), l’abbé de Saint-Pierre évoque une union des États de l’Europe afin de supprimer les guerres dans le monde. En 1756, le philosophe Jean-Jacques Rousseau rédige les Extraits et jugements sur le projet de paix perpétuelle, d’après les écrits de Saint-Pierre et note deux idées importantes : associer les États dans leurs relations extérieures, de la même façon que les citoyens sont associés dans l’État et par l’État dans une démocratie et créer un pacte protecteur de lois internationalement reconnues, auxquelles obéiraient les États signataires du pacte.
La Révolution française tenta sans grand succès d'unifier les peuples d’Europe contre les pouvoirs monarchiques. En 1795, dans son Essai sur la paix perpétuelle, Emmanuel Kant préconisait une « fédération d'États libres » :
« Ce serait là une « Fédération » de peuples, et non pas un seul et même État, l'idée d'État supposant le rapport d'un souverain au peuple, d'un supérieur à son inférieur. Or plusieurs peuples réunis en un même État ne formeraient plus qu'un seul peuple, ce qui contredit la supposition, vu qu'il s'agit ici des droits réciproques des peuples, en tant qu'ils composent une multitude d'États différents qui ne doivent pas se confondre en un seul\footnote{Paris : G. Fischbacher, 1880, p. 18-19}. »
En 1870, avant la Commune, les militants de l’Association internationale des travailleurs en France adressent un manifeste au peuple allemand : « tendons-nous la main, oublions les crimes militaires que les despotes nous ont fait commettre, les uns contre les autres. Proclamons : la liberté, l’égalité, la fraternité des peuples. Par notre alliance, fondons les États-Unis d’Europe\footnote{Discours du 1er mars 1871 à l'Assemblée nationale}. »
Mais ce fut surtout en réaction aux horreurs de la guerre qu'elle s'imposa avec plus de force, particulièrement après la guerre de 1870 : Victor Hugo appelait de ses vœux la construction d'un État paneuropéen, seul garant de la paix sur le continent. Aussi ces États-Unis d'Europe devaient être également l'endroit où les valeurs républicaines seraient respectées, afin que le monde entier s'en inspirât. (Voir dans Actes et Paroles « Paris »)
« […] Et on entendra la France crier : C'est mon tour ! Allemagne, me voilà ! Suis-je ton ennemie ? Non ! je suis ta sœur. Je t'ai tout repris, et je te rends tout, à une condition : c'est que nous ne ferons plus qu'un seul peuple, qu'une seule famille, qu'une seule république. Je vais démolir mes forteresses, tu vas démolir les tiennes. Ma vengeance, c'est la fraternité ! Plus de frontières ! Le Rhin à tous. Soyons la même République, soyons les États-Unis d'Europe, soyons la fédération continentale, soyons la liberté européenne, soyons la paix universelle ! »
Cette expression sera d'ailleurs reprise, dans les années 1920 par Aristide Briand. Un autre célèbre discours de Victor Hugo en faveur de la création d'une union du continent européen, est l'appel à l'union européenne en faveur de la Serbie pour l'aider dans sa lutte d'indépendance contre l'occupation turque.
Après la Première Guerre mondiale, la France imposa à l'Allemagne vaincue dans le traité de Versailles le paiement de réparations dans des conditions jugées humiliantes par les Allemands. On peut noter cependant les initiatives de Aristide Briand auprès de la SDN, ainsi que le mouvement « pan européen », véritable laboratoire d'idée pour le futur.
À cette période de nombreux marxistes mettent en avant l’objectif de créer des « États-Unis socialistes d’Europe », avec pour but d’empêcher les guerres et de franchir une étape pour aller vers un internationalisme véritable, tout en s’opposant au capitalisme.
Ce fut après les nouvelles horreurs de la Seconde Guerre mondiale que l'idée est reparue de façon plus large. Durant la guerre, des initiatives en vue de préparer la paix naquirent dans les mouvements de résistance. Ainsi, Altiero Spinelli fonde en 1943 à Milan le Mouvement fédéraliste européen. En 1944 est fondé à Lyon le Comité français pour la fédération européenne. En Allemagne, le mouvement La Rose blanche, dont les dirigeants seront exécutés par les nazis, appelle à une fédération européenne pour l'après-guerre (voir aussi : Fédéralisme européen).
Dans un célèbre discours prononcé en 1946 à l'université de Zurich, l'ancien Premier ministre britannique Winston Churchill déclarait :
« Il existe un remède qui, s'il était généralement et spontanément adopté par la grande majorité des peuples dans de nombreux pays pourrait, comme par miracle, rendre l'Europe aussi libre et heureuse que la Suisse de nos jours. […] Nous devons construire une sorte d'États-Unis d'Europe. […] La première étape consiste à former un Conseil de l'Europe. Et de ce travail urgent, la France et l'Allemagne doivent ensemble prendre la direction. […] Je vous dis donc : « Debout, l'Europe ! » »
En 1947, des socialistes de divers partis européens créent le Mouvement pour les États-Unis socialistes d’Europe. La même année, l’écrivain George Orwell se prononce également pour des États-Unis socialistes d’Europe.
Le Congrès de la Haye, en 1948, rassemble les différents partisans de l'unification de l'Europe. L'une de ses conséquences fut la création du Conseil de l'Europe, première tentative vers la construction d'un avenir commun aux nations d'Europe.
\chapter{Les Communautés européennes}

En 1946 à Cologne, Robert Schuman rencontre Konrad Adenauer et Alcide De Gasperi, respectivement chancelier allemand et président du conseil italien, lors d'une réunion entre les dirigeants des partis démocrates chrétiens. Lors de cette rencontre, ils exposent leurs idéaux et posent les fondements d'une Europe unie, dépassant l'opposition séculaire de certains peuples (notamment franco-allemands).
Le 9 mai 1950, Robert Schuman — alors ministre des Affaires étrangères — présenta au quai d'Orsay dans une déclaration, considérée comme l'acte de naissance de l'Union européenne, une proposition relative à une organisation de l'Europe, indispensable au maintien de relations pacifiques.
\section{La Communauté européenne du charbon et de l'acier : la première partie}
C'est ainsi que, suivant la proposition de Robert Schuman, la France, la République fédérale d'Allemagne, l'Italie, la Belgique, les Pays-Bas et le Luxembourg fondent la Communauté européenne du charbon et de l'acier (CECA). Le traité instituant cette organisation fut signé à Paris le 18 avril 1951, soit moins d'un an après la déclaration du 9 mai. D'une validité de 50 ans, il entra en vigueur le 23 juillet 1952.
Le traité instaure un marché commun du charbon et de l'acier, supprime les droits de douanes et les restrictions quantitatives entravant la libre circulation de ces marchandises et supprime toutes les mesures discriminatoires, aides ou subventions, qui seraient accordées par les États signataires à leur production nationale. La gestion de la production de la Communauté est placée sous le contrôle de la Haute Autorité, présidée initialement par Jean Monnet.
La mise en place du traité se fit en plusieurs étapes, avec la mise en place d'une période d'adaptation pour les industries nationales, et le marché ne fut totalement ouvert que le 18 février 1953 pour le charbon et le minerai de fer et le 1er mai 1953 pour l'acier.
\section{L'échec de la CED}
Alors que Robert Schuman, par le discours de l'horloge, annonce le projet de la CECA, le 9 mai 1950, commence le mois suivant, en juin 1950, la guerre de Corée. Les États-Unis exigent alors un réarmement rapide de l'Allemagne de l'Ouest. Jean Monnet propose l'idée d'une armée européenne, dans laquelle on pourrait intégrer la nouvelle armée Ouest-allemande. Dans son esprit, il s'agit d'abord de ne pas entraver les négociations en cours à propos de la CECA, mais aussi de saisir l'occasion pour créer rapidement une nouvelle structure fédérale européenne, comme le proposera, la CPE, Communauté politique européenne, qui devait établir une fédération des États européens. Le chef de gouvernement René Pleven, le 24 octobre 1950, dévoile devant l'Assemblée nationale, le projet conçu et écrit par Jean Monnet et son équipe. Le France propose donc à ses cinq partenaires européens une communauté européenne de défense.
Le projet initial ne convient, ni aux Américains, ni aux Allemands. De nombreuses négociations déconstruisent le projet initial de J. Monnet, si bien que le traité instituant la CED, n'est signé à Paris que le 27 mai 1952. Les États-Unis et la RFA ont réussi à modifier le traité, les premiers en obtenant par exemple, que les unités de base de cette armée européenne soient beaucoup plus importantes que dans le projet initial, les seconds obtiennent la reconnaissance de leur pleine souveraineté et le droit de constituer des divisions allemandes.
La ratification du traité est obtenue facilement en République fédérale Allemande, en Belgique, au Luxembourg et aux Pays-Bas. L'Italie attend le vote du Parlement français. En effet la ratification du traité de la CED fut en France, l'un des débats politiques les plus virulents de la quatrième République. « La querelle de la CED » sera si passionnée que plusieurs chefs de gouvernements cédistes (Pinay, Mayer, Laniel) ne prennent pas le risque de soumettre le traité au vote des députés. Il faut attendre Pierre Mendès-France, qui vient juste de conclure les accords de Genève, mettant fin à la guerre d'Indochine, pour qu'un chef de gouvernement ait le courage d'affronter l'épreuve de la ratification. Le 30 août 1954, le traité est rejeté par une question préalable adoptée par 319 voix contre 264. L'échec de la CED met fin à tout espoir d'intégration politique de l'Europe occidentale. Cependant, dès la conférence de Messine, en juin 1955, le projet d'intégration européenne est relancé dans le domaine économique.
\section{La CEE : du marché commun aux Communautés européennes}
Après l’échec de la ratification française de la CED, Jean Monnet démissionne avec fracas de la Haute autorité de la CECA ; il ne confie pas son prochain projet de coopération atomique européenne au gouvernement français, mais il s’approche du ministre belge des Affaires étrangères, Paul-Henri Spaak. Avec ses homologues hollandais et luxembourgeois, Spaak rédige le Mémorandum du Benelux (20 mai 1955), qui envisage la relance de la construction européenne sur deux volets complémentaires :
	\begin{enumerate}
		\item CEEA/Euratom, sur impulsion de Jean Monnet : une communauté européenne de l’énergie atomique, considérée à ce moment-là comme un remède-miracle contre la pénurie énergétique qui suit à la crise de Suez, et un objectif stratégique pour réduire la dépendance énergétique de l’Europe vis-à-vis de l’extérieur.
		\item Union douanière, sur impulsion de Ludwig Erhard : un marché commun doté de libre circulation des facteurs de production et de règles de concurrence, afin de garantir un marché pour les exportations belges et allemandes, et ne pas risquer une crise de surproduction (la France est sceptique, dans sa tradition de colbertisme).
	\end{enumerate}
À la Conférence de Messine du 1-3 juin 1955, les ministres des Affaires étrangères des pays de la CECA (Martino, Pinay, Hallstein, Bech, Beyen, Spaak) s’accordent pour une extension de l’intégration à tout secteur de l’économie. Le Comité Spaak (juillet 1955 - avril 1956) se charge de la rédaction d’un rapport. Le Royaume Uni se retire du comité en octobre, car il ne voit pas les bénéfices de l’union douanière, en préférant les rapports préférentiels avec le Commonwealth et la protection des secteurs industriels ; et non plus ceux de l’Euratom, grâce à son coopération renforcée en matière nucléaire avec USA et Canada. Le Rapport Spaak esquisse les caractéristiques de la CEE et de l’Euratom. Il prévoit une union douanière couplée par une tarif extérieur commun, dans l’optique d’un marché commun général ; une intégration sectorielle poussée est cependant prévue pour le secteur atomique, afin de partager les coûts relevant de la recherche et du développement du nucléaire.
Les mesures du Rapport Spaak sont discutées encore à la conférence de Venise de fin mai 1956, ou le gouvernement français (Pineau) remarque sa priorité pour Euratom (deux traités séparés) et la nécessité d’inclure l’association de la communauté économique avec les territoires d’outre-mer.
En octobre 1955, Monnet lance son Comité d’action pour les États-Unis d’Europe, afin de revitaliser et informer la société civile sur les développements de la politique européenne, et en gagner le soutien. Son mouvement préconise la formation d’un gouvernement commun, contrôlé par une assemblée à élection directe et universelle. Ils visent à éviter que les pressions britanniques arrivent à faire tomber le projet de marché commun.
La discussion internationale procède à la CIG de Val Duchesse, où les autres États craignent que la France puisse ne ratifier que le traité Euratom. Le marché commun est un grand souci pour le Benelux et l’Allemagne, qui voudraient une baisse des tarifs, tant à l'extérieur qu'à l'intérieur, tandis qu'Italie et France soutiennent des tarifs de protection extérieure. La France souligne l’importance de démarrer une Politique agricole commune et d’inclure les territoires d’outre-mer. Pour ce qui concerne l’Euratom, la France vise à garantir la possibilité d’utilisation militaire du nucléaire, tandis que les autres États ne sont pas intéressés (en 1954, l’Allemagne renonce volontairement à toute armement ABC) ; la discussion se termine avec la liberté des États de poursuivre leurs propres politiques nucléaires militaires, sous contrôle international.
Le 25 mars 1957 sont signés les Traités de Rome, qui entrent en vigueur après ratification le 1er janvier 1958. Ils ne contiennent aucune limite temporelle, ou procédure de retrait, et indiquent des finalités assez vagues (« une union sans cesse plus étroite entre les peuples »). Ils suivent le schéma fonctionnaliste de la CECA, mais avec moins de supranationalité : l’intérêt est de ne pas réveiller la querelle de la CED en France, qui a montré la difficulté d'accepter une extension des compétences supranationales. La Commission de la CEE (qui n’a pas le nom, politiquement chargé, de Haute autorité), garde seulement le pouvoir d’initiative et de gardienne des Traités, tandis que la majorité des pouvoirs sont confiés au Conseil de la CEE. L’Assemblée des délégués et la Cour de Justice sont communes pour les trois communautés.
L’association des territoires d’outre-mer est garantie par un accord quinquennal de coopération commerciale (zone de libre-échange entre les territoires et chacun des États de la CEE) et financière (fonde FED de coopération). L’intérêt français de compenser la perte des liens politiques (guerre d’Algérie) avec plus de liens commerciaux est satisfait, en évitant le risque de tomber dans le néo-colonialisme condamné par l’ONU.
\subsection{La naissance de la CEE}
Le 25 mars 1957, les six pays décident en signant le Traité de Rome d'aller plus loin dans la coopération. Tous les domaines économiques, mais aussi politiques et sociaux sont concernés. Le but est d'aboutir économiquement à un marché commun, le marché commun européen. La Communauté économique européenne (CEE) est l'entité internationale créée par le traité de Rome.
Le Traité de Rome de 1957 prévoit la mise en place d'une union douanière au sein de laquelle sont abolis droits de douane, restrictions quantitatives (quotas) et les mesures d'effets équivalents à des droits de douane. Cette union adopte un tarif extérieur commun pour les marchandises en provenance d'État tiers. Cette libre circulation des marchandises entre les Six consacre la vision de Monnet d'une intégration fonctionnelle : la création de telles solidarités de fait dans le domaine économique doit amener à des interactions entre les États européens forçant à terme une intégration politique (théorie du functionnal spill over ou de l'effet d'entraînement).
Outre la libre circulation est prévue la mise en place de trois politiques communes, supranationales en ce que la Communauté se voit attribuer une compétence exclusive dans ces 3 domaines (transport, agriculture, politique commerciale, cette dernière résultant naturellement de la mise en place du tarif extérieur commun).
La commission européenne est créée et est chargée de créer une administration intégrée pour veiller à la mise en place des objectifs du Traité. Elle ne dispose cependant que d'un pouvoir de proposition, les décisions devant être prise par le Conseil européen (réunion des ministres des États membres dans leur domaine). Le Parlement européen n'ayant, quant à lui, qu'une fonction consultative.
Les institutions semblent menacées dès leur naissance : l'agonie de la IVe république française amène le général de Gaulle au pouvoir. De Gaulle est très attaché à l'indépendance nationale, ce qui ne l'empêche pas de croire à un destin mondial de l'Europe. Doutant des possibilités de l'intégration européenne, et considérant que la nation est le ciment de l'État moderne, il préfère l'entente entre les nations (conférence de presse du 30 décembre 1961). Charles de Gaulle n'acceptera les institutions communautaires que dès lors qu'elles seront compatibles avec son projet pour la France. En témoigne le rejet européen du plan Fouchet en 1962 (plan pour la création d'une Union politique mais de type plus intergouvernemental que le projet de CED) proposé par le général (il s'agit en fait de diminuer les pouvoirs de la Commission européenne ainsi que ceux du Parlement européen). Il essuiera un refus net de la part des autres pays membres. Au cours d'une conférence de presse donnée à Paris, le 14 janvier 1963, le président de la République française définit les grands axes de la politique étrangère du pays. Il exprime en premier lieu sa ferme opposition à l'entrée de la Grande-Bretagne dans le marché commun européen (CEE). Le chef de l'État justifie sa position en affirmant que « la nature, la structure, la conjoncture, qui sont propres à l'Angleterre diffèrent de celles des États continentaux ». Il se méfie surtout des solides relations que le Royaume-Uni entretient avec les États-Unis d'Amérique. Dans la même conférence de presse, il réaffirmera d’ailleurs l'autonomie de la défense nucléaire française face aux États-Unis. Bien que ses partenaires y soient favorables, il mettra une nouvelle fois son veto à l’entrée de la Grande-Bretagne dans le CEE en 1967.
Souhaitant effacer les blessures des deux guerres mondiales, le chancelier Konrad Adenauer et le général de Gaulle entendent fonder l'Europe politique sur une coopération entre les deux pays :
« Deux grands peuples, qui se sont longuement et terriblement opposés et combattus, se portent maintenant l’un vers l’autre dans un même élan de sympathie et de compréhension. Il ne s’agit pas seulement d’une réconciliation commandée par les circonstances. Ce qui se produit, en vérité, c’est une espèce de découverte réciproque des deux voisins, dont chacun s’aperçoit à quel point l’autre est valable, méritant et attrayant.
De là, part ce désir de rapprochement manifesté partout dans les deux pays, conforme aux réalités et qui commande la politique parce que, pour la première fois depuis maintes générations, les Germains et les Gaulois constatent qu’ils sont solidaires. »
Ils vont donc signer le traité de l'Élysée (1963), inaugurant la réconciliation franco-allemande ouvrant une période d'amitié entre les deux peuples, qui se poursuivra avec tous les chefs d'État ultérieurs des deux pays. Le couple franco-allemand devient alors « le moteur de l'Europe ». Toutefois, le traité sera largement vidé de sa substance par l'introduction d'un préambule atlantiste voté par le Bundestag sous l'influence de Jean Monnet et des États-Unis :
« Convaincu que le traité du 22 janvier renforcera et rendra effective la réconciliation et l’amitié, il constate que les droits et les obligations découlant pour la République fédérale de traités multilatéraux ne sont pas modifiés par ce traité, il manifeste la volonté de diriger l’application de ce traité dans les principaux buts que la République fédérale poursuit depuis des années en union avec les autres alliés, et qui déterminent sa politique, à savoir, maintien et renforcement de l’alliance des peuples libres et, en particulier, étroite association entre l’Europe et les États-Unis d’Amérique ; défense commune dans le cadre de l’Alliance atlantique ; unification de l’Europe selon la voie tracée par la création de la Communauté en y admettant la Grande-Bretagne, renforcement des communautés existantes et abaissement des barrières douanières. »
Le traité n'aura d'implication que sur le plan culturel, avec la création de l'Office franco-allemand pour la jeunesse.
Le Président De Gaulle n'entend pas entrer dans la logique fédérale et supranationale et précise son point de vue dans une conférence de presse du 14 décembre 1965 :
« Je crois qu’il y a, dès lors que nous ne nous battons plus entre Européens occidentaux, dès lors qu’il n’y a plus de rivalité immédiate, et qu’il n’y a pas de guerre, ni même de guerre imaginable, entre la France et l’Allemagne, entre la France et l’Italie, et même, bien entendu, un jour, entre la France, l’Italie, l’Allemagne et l’Angleterre... et bien il est absolument normal que s’établisse entre ces pays occidentaux une solidarité. C’est cela l’Europe ! Et je crois que cette solidarité doit être organisée : il s’agit de savoir comment et sous quelle forme. Alors, il faut prendre les choses comme elles sont, car on ne fait pas de politique autrement que sur des réalités. Bien entendu, on peut sauter sur sa chaise comme un cabri en disant l’Europe ! l’Europe ! l’Europe ! Mais cela n’aboutit à rien et cela ne signifie rien. »
Des difficultés survinrent avec la crise dite « de la chaise vide » en 1965. Le président de la Commission, l'Allemand Walter Hallstein, propose un nouveau mode de financement de la PAC, c’est-à-dire que la Communauté, au lieu de redistribuer les contributions des États Membres, collecterait les droits de douane issus du tarif extérieur commun pour financer la PAC. Les instances communautaires seraient ainsi en mesure de disposer de fonds propres dont l'utilisation aurait été soumise au contrôle du Parlement européen. La France rejeta cette proposition et se saisit de l'occasion pour remettre en question le principe du vote au conseil à la majorité qualifié à la fin de la période de transition (c’est-à-dire en 1970), elle entame ainsi la politique de la chaise vide et le veto systématique qui en résulte bloque les institutions. Le compromis de Luxembourg nécessaire pour y mettre fin dispose que « lorsque, dans le cas de décision susceptible d'être prise à la majorité sur proposition de la Commission, des intérêts très importants d'un ou plusieurs partenaires sont en jeu, les membres du Conseil s'efforceront, dans un délai raisonnable, d'arriver à des solutions qui pourront être adoptées par tous les membres du Conseil, dans le respect de leurs intérêts et de ceux de la Communauté ».
Le principe du vote à la majorité est sérieusement limité, renforçant ainsi la logique intergouvernementale au détriment de l'approche supranationale. La méthode dite de synchronisation adoptée comme méthode de travail au sein du Conseil sur proposition du ministre Allemand Schroeder le 1er avril 1963, et qui consiste à isoler les points de désaccord entre membres du Conseil et à les résoudre selon le principe de réciprocité renforçant davantage cette approche intergouvernementale : les conflits sont résolus en fonction des intérêts des États membres uniquement.
Le Traité de fusion est signé le 8 avril 1965 à Bruxelles ; il fusionne les exécutifs des trois communautés, qui à l'origine partageaient déjà la Cour de Justice et le Parlement. Ce sont donc les Conseils et Commissions (appelée Haute Autorité dans le cas de la CECA) qui sont regroupés en un seul Conseil et une seule Commission, basés à Bruxelles. Ce traité entre en vigueur le 1er juillet 1967.
Alors que la France refuse de prendre part au Conseil européen depuis juillet 1965, tous les membres de la CEE se rassemblent, le 29 janvier 1966 à Luxembourg, pour trouver un compromis et mettre fin à la crise. La France reprochait notamment à la Communauté de n’avoir pas tenu ses engagements quant au financement de la Politique agricole commune (PAC). Au terme des discussions, plusieurs compromis seront adoptés, dont le principe d’unanimité décisionnelle. Un État membre pourra désormais faire reporter un vote et prolonger les discussions sur un projet, dans la mesure où celui-ci affecte ses intérêts essentiels.
La France met une nouvelle fois son veto à l’entrée de la Grande-Bretagne dans le CEE en 1967.
Article détaillé : Relance de La Haye.
La fin des années soixante voyait le blocage des propositions d’intégration européenne par De Gaulle :

	\begin{description}
		\item [en 63 :] le traité de l'Élysée pour la concertation en politique extérieure avec l’Allemagne ;
		\item [en 63 et 67 :] les deux vétos à l’adhésion britannique à la CEE ;
		\item [en 65-66 :] la crise de la chaise vide, contre la volonté de Walter Hallstein de faire voter le Conseil à majorité qualifiée ;
		\item [en 69 :] la tentative de directoire franco-britannique (affaire Soames) contre la CEE.
	\end{description}
Il apparaît clair que la relance européenne devait passer par un changement à la tête du gouvernement français. Cela se passait avec Pompidou en 1969, doublé par l’arrivée aussi de Heath et Brandt. Pompidou, bien que gaulliste, avait été élu sur une plateforme pro-européenne, et avait bien clair que la CEE devait répondre aux demandes britanniques ; Heath, conservateur, proposait tout de suite une troisième candidature en 1970. Les 1er et 2 décembre 1969, au sommet de la Haye sous présidence hollandaise, les chefs de gouvernement de la CEE s’accordent sur un programme de réforme. La conférence se clôt sur plusieurs décisions qui donnent une nouvelle dynamique à la construction communautaire :
	\begin{itemize}
		\item achever :
		\item approfondir :
		\item élargir : le principe de l'élargissement est accepté par la France qui obtient l'obligation d'une entente préalable entre les Six sur les conditions d'adhésion. Avec la levée du véto français, on peut démarrer les négociations pour l’élargissement au Royaume Uni, l’Irlande, le Danemark et la Norvège. Pompidou a son intérêt à avoir le Royaume Uni dans la Communauté, et à son côté, pour mieux répondre et résister aux impulsions fédéralistes et supranationalistes de l'Allemagne et du Benelux.
	\end{itemize}
\section{Du premier élargissement à l'Union européenne}
	\begin{description}
		\item [1er janvier 1973 :] Royaume-Uni, Danemark et Irlande ;
		\item [1er janvier 1981 :] Grèce ;
		\item [1986 :] Espagne et Portugal
	\end{description}
\subsection{L'élargissement au Nord}
Le premier élargissement de la CEE aux années soixante-dix constitue une conséquence de la bonne réussite économique de l’union douanière, achevée par la CEE en 1968, par rapport à l’intégration lâche de l’EFTA/AELE (zone de libre-échange). La CEE pouvait en effet compter sur une majeure taille, une contigüité géographique, et les hauts taux de croissance économique de la période de la reconstruction. Déjà en 1961 le Royaume Uni soumette demande d’adhésion, mais De Gaulle met le véto français en 1963 comme en 1967, contre l’avis favorable des autres cinq États membres.
Les conditions se modifient en 1969/1970, avec l’arrivée au gouvernement de Heath, Pompidou et Brandt. Heath soumet une troisième candidature, à laquelle la France ne met pas son véto. Les négociations d’adhésion commencent déjà en 1970. Pour le Royaume Uni on envisage un période d'adaptation à l’acquis communautaire, à cheval de l’adhésion, et sept ans de transition avant de l’intégration de la PAC, pour régler les rapports avec les pays du Commonwealth, tandis que pour Danemark, Irlande et Norvège on prévoit l’accès direct aux fonds PAC et 5 ans d’adaptation à l’acquis. Pour le Royaume Uni, l’adhésion à la CEE représentait la possibilité de revitaliser son économie et de contrer la croissante influence française/gaulliste dans le continent (sous pression américaine), outre que de chercher une nouvelle politique extérieure suite à l’échec impériale de Suez 1956.
Entre 1961 (première candidature) et 1973 (adhésion) les performances économiques de la CEE même avaient suivi une tendance négative, et l’adhésion du Royaume Uni pouvait représenter une impulsion économique. Par contre, Londres craignait désormais d’être obligé à une position de contributeur net dans la CEE, étant donné que son agriculture n’aurait pas attiré autant de fonds PAC. Pour cette raison, son entrée dans la CEE pousse à l’introduction de la politique régionale de cohésion, en mesure de bénéficier les régions les plus marginales (avec soutien de l’Italie). Du côté législatif, le fair play fit ainsi que la Communauté atteint l’entrée du UK avant de proposer des sauts qualitatifs dans l’acquis. Les mêmes problèmes ne se présentèrent pas pour les autres pays nordiques intéressés par l’adhésion, dont les économies étaient strictement liées à la Grand Bretagne, et qui auraient bien bénéficié des fonds PAC. Toutefois, l’adhésion fut rejetée par referendum en Norvège, car les règles européennes auraient impacté fort sur le secteur de la pêche
L’élargissement du 1973 renforce l’axe économique Londres-Rotterdam-Rhin, et représente une ouverture de la CEE vers le Nord, à tradition protestante, libérale et atlantiste, en contraste avec les pays du sud (France et Italie), plus dirigistes. À niveau économique, la nécessité d’harmoniser les tarifs extérieurs des nouveaux pays membres amène à un accord de libre-échange CEE-AELE (ce qui n’avait pas été possible dans les années 1960).
En 1974, l’arrivée au gouvernement anglais du laboriste Harold Wilson amène le Royaume Uni à demander une renégociation des termes d’adhésion, soutenu par la menace d’un referendum populaire sur la CEE. Le Royaume Uni obtient des concessions par rapport au régime des importations (convention de Lomé 1975) qui substitue Yaoundé II, rattachant les colonies britanniques dans la définition de pays ACP), la mise en œuvre de la politique régionale, et la possibilité (bien que remote) d’un remboursement des contributions anglaises.
Le 22 janvier 1972 à Bruxelles, l'Irlande, le Royaume-Uni, le Danemark et la Norvège signent un traité d'adhésion au marché commun européen. Dès le 1er janvier 1973, les Britanniques, les Danois et les Irlandais intègrent la CEE ; en revanche, les Norvégiens refuseront par référendum d'entrer dans la Communauté européenne.
\subsection{Le système monétaire européen}
Réuni à Paris, le 13 mars 1979, le Conseil européen prend la décision de créer un système monétaire européen, le SME. La nouvelle monnaie européenne, qui n'est, dans un premier temps, qu'une unité de compte, est baptisée ECU.
La relance européenne de 1969, avec la coopération entre Pompidou et Helmut Schmidt, amène entre autres à la rédaction du plan Werner sur la coopération monétaire. Le plan envisageait une convertibilité fixe garantie entre monnaies européennes pour l’années 1980, afin de contrer les effets de distorsion du marché commun causés par les fluctuations monétaires. Le plan s’écroule avec la crise du système de Bretton Woods en 1971, et les réponses nationales protectionnistes à la crise monétaire. Pour limiter les distorsions du marché commun causé par les fluctuations monétaires, les États s’accordent sur le « tunnel monétaire » (Accords de Washington, 1971, 10\% sur le dollar) en sens externe, et sur le serpent dans le tunnel (1971, 5\% fluctuation réciproque, puis 2,25\%) en sens interne. Le système reste très instable et sujet aux spéculations internationales.
En 1977 Roy Jenkins, président britannique de la Commission, à Florence, propose la création du Système Monétaire Européen (SME), qui entre en vigueur en 1979. Le SME se constitue come système de fluctuations concordées sur la moyenne de l’ECU ; il prévoit aussi l’indépendance des Banques Centrales, un Fonds européen de coopération monétaire (FECOM) et la convergence des politiques économiques. Le SME traverse des phases de turbulence, consolidation et stabilité, grâce à la place centrale acquise par le Deutsche Mark comme monnaie d’ancrage, jusqu’à son écroulement en 1992.
Dans les années 1980, la contribution des industrialistes de l’ERT (Table ronde des Industriels européens) souligne le manque de connexion et les barrières non tarifaires qui segmentent le marché commun, avec l’effet d’en réduire les bénéfices. Le nouveau but devient le marché unique (Rapport Delors 1985 et Acte Unique 1986), à soutenir par une monnaie commune. Le Sommet d’Hanovre en 1988 donne l’aval, et en 1989 Delors présente dans son rapport un plan en trois étapes, approuvé au conseil de Madrid 1989 et demandé au groupe Guigou pour détails. Au sommet de Strasbourg du décembre 1989, on approuve définitivement, et le plan rentre dans le texte de l’accord de Maastricht de '91/'93. L’introduction d’une monnaie unique, au lieu d’autres alternatives (taux d’échange fixes garantis, pour l’instance) est une décision politique avant d'être économique : elle représente, avec la coopération CFSP en politique extérieure, une multilatéralisation de la politique allemande, considérée par Kohl un prix à payer (déjà en 1988, la renonce aux symboles de la puissance économique) pour permettre l’aval des autres États européens à la réunification soudaine. Par contre, les contraints forts des critères de Maastricht, poussés par l’Allemagne, représentent la volonté que l’euro soit aussi fort que le mark, et la BCE aussi indépendant que la Bundesbank.
Les plans d’intégration monétaires doivent faire face à la crise économique du 1992, mais arrivent à bien procéder grâce à la reprise économique mondiale dans la deuxième partie de la décennie. En 1999 la vérification des critères amène 11 pays (puis 12) à rentrer dans la zone euro dès le début. La circulation de la nouvelle monnaie prend pied en 2002.
\subsection{L'élection directe du Parlement}
Depuis le 7 juin 1979, les citoyens des neuf États membres de la Communauté européenne élisent pour la première fois les députés du Parlement européen au suffrage universel direct. La plus forte participation est celle de la Belgique avec 91\% (le vote est obligatoire en Belgique) et la plus faible celle de la Grande-Bretagne, avec 31\%. En France, elle s'élève à 60\%. Le Parlement, dont le siège est à Strasbourg, a un rôle consultatif. Mais il est également compétent pour légiférer aux côtés du Conseil des ministres et exerce un contrôle sur la Commission. Élu en juin au suffrage universel, le Parlement européen siégeant à Strasbourg procède, lors de sa première session, le 17 juillet 1979, à l'élection de son président. À la majorité absolue et au deuxième tour, l'ancienne ministre française de la Santé, Simone Veil (52 ans), l'emporte. Madame Veil, qui conduit la liste UDF (Union pour la démocratie française), le parti du président Valéry Giscard d’Estaing, est connue du grand public pour son combat en faveur de la légalisation de l'interruption volontaire de grossesse en 1975. Élue pour cinq ans à la présidence du Parlement européen, elle s'attachera jusqu'en 1982 à promouvoir l'élargissement de l'Europe tout en ayant à cœur d'améliorer les conditions sociales des Européens.
\subsection{L'élargissement au Sud}
Le 1er janvier 1981, la Grèce intègre la Communauté économique européenne (CEE) et devient le pays le plus pauvre de la communauté, avec une inflation et un chômage catastrophique.
La France s'oppose d'abord à l'entrée de l'Espagne et du Portugal dans la CEE. En effet, les partis communiste et gaulliste (RPR) voient dans ces pays de féroces concurrents agricoles. Néanmoins, un compromis est trouvé et les pays candidats se voient imposer des quotas de vente. Le Portugal et l'Espagne signent, le 12 juin 1985, leur adhésion à la Communauté économique européenne. La CEE comptera donc désormais douze pays et 320 millions d'habitants, sur une superficie de 2 millions de km². Des Programmes intégrés méditerranéens (PIM) sont mis en place pour permettre un rattrapage des nouveaux adhérents (Portugal, Espagne, Grèce ; France et Italie étant aussi éligibles) en termes d'infrastructures et faciliter ainsi leur intégration.
\subsection{Schengen, l'Acte Unique et la route vers Maastricht}
Le 30 novembre 1979, le Premier ministre britannique Margaret Thatcher demande « un rabais » de la contribution britannique au budget européen. Elle réussit à faire valoir ses prétentions le 26 juin 1984 au Conseil européen de Fontainebleau. C'est ce qu'on appelle depuis le « chèque » britannique.
Les accords signés à Schengen (Luxembourg) par plusieurs États européens, le 14 juin 1985, prévoient l'abolition des contrôles aux frontières communes entre les États signataires. Cette suppression des contrôles intérieurs est accompagnée de la mise en place de règles communes sur l'entrée et le séjour des ressortissants n'appartenant pas à la Communauté européenne. Ces accords seront complétés par une convention d'application en 1990 et entreront en vigueur en 1995.
Signé en février 1986, le traité de l’Acte unique européen entre en vigueur le 1er juillet 1987. Il apporte des modifications au traité de Rome, et donc, à la Communauté économique européenne (CEE). Son objectif est en fait d’accélérer la mise en place du marché intérieur, dont l’achèvement est prévu pour décembre 1992. Il est, pour cela, nécessaire de renforcer les pouvoirs des institutions européennes (Conseil, Parlement, Commission) et d’élargir leurs domaines de compétence à l’environnement, à la politique étrangère et à la recherche technologique.
Avec le traité de Maastricht signé en 1992, la CEE sera intégrée dans l'Union européenne (UE). Elle est renommée Communauté européenne le 1er novembre 1993.
\chapter{Naissance de l'Union européenne}

\section{Traité de Maastricht}
	\begin{description}
		\item [1992 :] Sommet de Maastricht ;
		\item [1993 :] entrée en vigueur du traité de Maastricht ;
		\item [1995 :] adhésion de la Suède, de l'Autriche et de la Finlande, en application du traité de Corfou ;
		\item [1997 :] traité d'Amsterdam ;
		\item [2000 :] traité de Nice.
	\end{description}
Le Traité de Maastricht représente une nouveauté, en tant qu’il envisage une union politique à dimension étatique. Il constitue l’approfondissement de la gouvernance européenne, avant de s’élargir aux pays neutres et postcommunistes. Plusieurs caractéristiques de la nouvelle structure de Maastricht représentent un saut qualitatif dans la construction européenne :
	\begin{enumerate}
		\item l’introduction de nouveaux champs de compétence (CFSP et JAI), bien que seulement à niveau de coopération intergouvernementale, parallèle à la méthode communautaire de la CE ;
		\item les attributions étatiques de la nouvelle UE qui regroupe CE, CFSP et JAI : les éléments constitutionnels prévus dans l’implant des Traités, et la définition de citoyenneté européenne ;
		\item la réforme de la gouvernance, et le rôle accru du Parlement Européen ;
		\item l’objectif de l’union monétaire comme prochaine étape de l’intégration économique après le marché unique. Cela, couplé avec la coopération en politique extérieure (CFSP), marque une double multilatéralisation de la politique allemande, prix à payer pour la réunification soudaine.
	\end{enumerate}
De l’autre côté, l’implant du Traité de Maastricht reste incomplet :
	\begin{enumerate}
		\item le résultat est asymétrique, car les nouvelles compétences de l’Union sont soumises aux vétos des États ; les accords sont ensuite révisés à Amsterdam (1997) et Nice (2000) ;
		\item il surgit un capability/expectation gap en CFSP (trop d’attentes pour trop peu de possibilités réelles en politique étrangère), qui amène à une crise de confiance avec l’éclatement des guerres yougoslaves ;
		\item les plans d’intégration monétaires doivent faire face à la crise économique du 1992, mais arrivent à bien procéder grâce à la reprise économique mondiale dans la deuxième partie de la décennie ;
		\item l’approfondissement des compétences et la nouvelle relevance par rapport aux États amènent à une crise de légitimité. On commence à parler d’un déficit démocratique de l’Union, causé par le manque d’intégration des masses dans le système politique communautaire.
	\end{enumerate}
\section{Union économique et monétaire}
	\begin{description}
		\item  [1er janvier 1999 :] naissance de l'euro (au cours de 1 euro = 1 ECU) et adoption par onze pays membres ;
		\item [1er janvier 2001 :] la Grèce adopte à son tour l'euro ;
		\item [En juin 2001 :] les Irlandais votent « non » au référendum sur le traité de Nice et l'élargissement. Après une mise au point des politiques, les Irlandais ont finalement ratifié le traité à l'occasion d'un nouveau référendum, en octobre 2002.
		\item [1er janvier 2002 :] introduction des pièces et billets en euro ;
		\item [1er trimestre 2002 :] suppression du cours légal des monnaies nationales dans les pays ayant adopté l'euro.
		\item [1er janvier 2007 :] la Slovénie rejoint la Zone euro.
		\item [1er janvier 2008 :] Chypre et Malte rejoignent la Zone euro.
		\item [1er janvier 2009 :] la Slovaquie rejoint la Zone euro.
		\item [1er janvier 2011 :] l'Estonie rejoint la Zone euro.
	\end{description}
Cet événement a traumatisé les institutions européennes. Après qu'une nouvelle majorité fut issue des urnes (élections présidentielles puis législatives de 2007), une nouvelle forme du même traité, dit Traité modificatif, fut lancé et permit à l'Union européenne de sortir de la crise (mais sans que les peuples ne soient de nouveau consultés pour éviter le risque d'un rejet). Toutefois, la consultation du peuple est constitutionnellement obligatoire en Irlande.
\section{De nouveaux enjeux : la gouvernance d'internet}
La Commission européenne, en vertu du monopole d'initiative qu'elle exerce dans le cadre de ses compétences sur le premier pilier de l'Union européenne, est force de proposition à travers les comités consultatifs qui interviennent dans les processus de décision, tout particulièrement pour ce qui touche à la communication par l'internet et aux Livres blancs, tous sujets qui touchent à la politique européenne de développement durable.
\section{Bulgarie et Roumanie}
Membres depuis le premier janvier 2007, l'adhésion de la Bulgarie et de Roumanie à l'Union avait été confirmée le 26 septembre 2006 par le président de la Commission. La commission a décidé de ne pas exercer le report de l'adhésion (d'un an) mais conformément aux dispositions du traité d'adhésion, elle surveillera rigoureusement (durant trois ans) trois points controversés pour ces deux nouveaux membres en se réservant le droit de suspendre l'application de certaines politiques européennes s'ils ne prennent pas les mesures nécessaires dans les domaines où de graves défaillances auront été constatées :
	\begin{itemize}
		\item la justice et les affaires intérieures,
		\item le marché intérieur,
		\item les échanges commerciaux.
	\end{itemize}
Cette situation placera la zone tampon de la Transnistrie, État non reconnu des instances internationales et identifié comme particulier pour ses licences douanières, à 200 km des frontières de l'Union européenne.
\chapter{Penseurs et acteurs de l'Union européenne}

\section{Avant la Seconde Guerre mondiale}
	\begin{description}
		\item [L'empereur Charlemagne :] surnommé le « père de l'Europe » par un poète anonyme du IXe siècle.
		\item [Érasme :] humaniste, théologien \footnote{La théologie est l'étude et l'exégèse de la religion, de Dieu, des textes sacrés ou des dogmes} et précepteur néerlandais, l'un des plus représentatifs de la Renaissance européenne.
		\item [Luis Vives :] humaniste espagnol du XVIe siècle.
		\item [Charles-Irénée Castel de Saint-Pierre (1658 - 1743) :] dit l'abbé de Saint-Pierre, écrivain et diplomate français ; auteur du Projet pour rendre la paix perpétuelle en Europe.
		\item [Jean-Jacques Rousseau :] philosophe de langue française du XVIIIe siècle célèbre de son vivant dans toute l'Europe.
		\item [Friedrich Schiller :] compositeur allemand, auteur en 1785 de l'Ode à la joie.
		\item [Emmanuel Kant :] philosophe allemand du XVIIIe siècle.
		\item [Saint-Simon :] publie, avec l'aide d'Augustin Thierry, De la réorganisation de la société européenne dans lequel il propose une réconciliation entre la France et l'Angleterre.
		\item [Napoléon Ier :] général, homme politique et empereur des français de 1804 à 1815.
		\item [Victor Hugo :] poète et célèbre écrivain français, auteur du livre Les Misérables.
		\item [Jean Jaurès :] normalien, agrégé de philosophie. Écrivain, homme politique et député socialiste français assassiné en 1914 à cause de son pacifisme.
		\item [Romain Rolland :] écrivain, humaniste et pacifiste français.
		\item [Richard Nikolaus de Coudenhove-Kalergi :] qui publie Paneuropa dans lequel il propose le premier projet moderne d'une Europe unie.
		\item [Gustav Stresemann (1878 - 1929) :] homme politique et diplomate allemand, prix Nobel de la paix en 1926.
		\item [Aristide Briand :] président du conseil et ministre français.
	\end{description}
\section{Les Pères fondateurs}
	\begin{description}
		\item [Jean Monnet :] économiste français, président-fondateur de la CECA.
		\item [Robert Schuman :] résistant, ministre français et président du Parlement européen.
		\item [Konrad Adenauer :] homme politique allemand, chancelier de 1949 à 1963.
		\item [Winston Churchill :] premier ministre britannique de 1940 à 1945, puis de 1951 à 1955.
		\item [Altiero Spinelli :] homme politique italien et fondateur du Mouvement fédéral européen.
		\item [Alcide De Gasperi :] président du Conseil italien et président du Parlement européen.
		\item [Paul-Henri Spaak :] premier ministre belge et président du Parlement européen.
	\end{description}
\section{Les acteurs successeurs}
	\begin{description}
		\item [Albert Camus :] écrivain français
		\item [Charles de Gaulle :] président de la République française de 1958 à 1969.
		\item [Willy Brandt :] chancelier allemand de 1968 à 1974.
		\item [Georges Pompidou :] président de la République française de 1969 à 1974.
		\item [François-Xavier Ortoli :] ministre français et Président de la Commission européenne de 1972 à 1977.
		\item [Valery Giscard d'Estaing :] président de la République française de 1974 à 1981 et Président de la Convention européenne de 2001 à 2004.
		\item [Simone Veil :] ministre française et Présidente du Parlement européen.
		\item [François Mitterrand :] président de la République française de 1981 à 1995. Artisan du Traité de Maastricht.
		\item [Helmut Kohl :] chancelier allemand de 1982 à 1998. Artisan du Traité de Maastricht.
		\item [Jacques Delors :] ministre français et Président de la Commission Européenne de 1984 à 1995. Artisan de l'euro.
		\item [Edgar Morin :] philosophe, sociologue et essayiste français. Directeur de recherche émérite au CNRS. Pour l'élaboration d'un dessein européen commun.
	\end{description}


\end{document}
